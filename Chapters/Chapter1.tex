
% Chapter 1

\chapter{Introduction} % Main chapter title

\label{Chapter1} % For referencing the chapter elsewhere, use \ref{Chapter1} 

%----------------------------------------------------------------------------------------

% Define some commands to keep the formatting separated from the content 
\newcommand{\keyword}[1]{\textbf{#1}}
\newcommand{\tabhead}[1]{\textbf{#1}}
\newcommand{\code}[1]{\texttt{#1}}
\newcommand{\file}[1]{\texttt{\bfseries#1}}
\newcommand{\option}[1]{\texttt{\itshape#1}}

%----------------------------------------------------------------------------------------

\section{Background}
Preterm birth (births before 37 completed weeks of gestation) is the major cause of neonatal deaths in the world \textcite{Arnold2013}. Although improvements in neonatal care have increased survival for preterm infants, prevention research and knowledge to date have not managed to reduce the rate of premature births. The global burden of preterm birth is estimated to be 15 million with substantially higher rates in developing countries \textcite{Steyn2017, Blencowe2013, Althabe2012}. Many developing nations  have an increasing focus on the neonatal care with tertiary level centers being established to provide neonatal training programs \textcite{Lloyd2013a, Enlow2017}. 

However, despite remarkable progress in neonatal care, little work has been done is supporting parents of premature infants. Parenting an infant in the neonatal intensive care unit (NICU) comes with a multitude of unique challenges, and NICU parents are often unprepared and ill equipped for the challenges \textcite{Howe2014}. Studies have reported that parents of preterm infants -- especially the mothers experience, increased level of stress in the neonatal period and they are more likely  to suffer from depression and anxiety at the time of hospital discharge \textcite{Guillaume2013, Wigert2014b, Heidari2017}. Therefore, stress management is very important during infant hospitalization. 

Various studies recommend Family-Centered Care (FCC) care model as the "best practice" in NICU in both developed and developing countries \textcite{Al-Motlaq2017, Rostami2015}. FCC involves holistic care and requires cooperation between parents and NICU staff in the care of the hospitalized infants. However, a gap still exists concerning holistic involvement of parents in premature infants care. Most parent lack full understanding of their parental role in the NICU. Moreover, they feel their psychosocial needs are neglected thus they experience anxiety due to the vulnerable state of their infants' health. In these cases, most parents rely on NICU staff for support during their infants' hospitalization.

\section{Communication between Parents and Neonatal Intensive Care Unit Staff}
Communication between parents and NICU staff is an essential part of the support offered to the parents and can reduce their emotional stress \textcite{Turner2015, Jones2015, Orzalesi2011, Kowalski2006}. Parents want clear and honest information, and most of them have challenges in obtaining accurate and up-to-date information from the NICU staff. The reason is that most NICU staff have heavy work load and they mainly focus on ensuring the health status of the infants stabilizes. In addition, NICU staff lack adequate and efficient training to support them as they interact with the parents. Orzalesi et al. \textcite{Orzalesi2011} alluded that most staff lack definite answers required by parents particularly with regard to the prognosis of the infants thus compromising parents' trust on the NICU staff.

This situation, is worse in developing world context where there is staff shortage in the unit. A study by Richter et al. \textcite{Ritcher2009} found that little thought is given to the role of parents when their sick infants is admitted to the NICU and planning for the unit does not include the parents. Instead, parents are mainly involved in giving information about their sick infant and preparing the infant for procedures but not in planning and implementing care. Moreover, most parents are faced with numerous challenges which limit them from visiting the unit frequently. This limitations include lack of transportation cost,family obligations, job and income loss, social stigma and cultural beliefs \textcite{Martinez2012, Thompson1993, Heidari2012}.

This results in the separation of parents from their infants and they feel they have lost the parental role to NICU staff. To follow up on their infants progress, the parents use phone calls to contact the NICU staff. However, these modes of communication are expensive because parents experience long delays before their calls are transferred from the main hospital switchboard to the NICU extension \textcite{Mburu2018}. Furthermore, parents are unreachable when NICU staff use phone call to reach out to because some parents provide incorrect phone numbers or they use shared phone in their household.

\section{Use of Technology to Support Parents of Preterm Infants}
In recent years, a nascent body of research has emerged, exploring  how digital technologies could be used to inform, assist, empower, and support parents of premature infants in the NICU and afterwards. Interventions such as Baby CareLink \textcite{Gray2000}, My Preemie \textcite{Doron2013}, Estrellita \textcite{Hayes2014}, NICU-2-HOME \textcite{Garfield2014}, Baby Talk \textcite{Mahamood2011} and the internet-based program \textcite{Lindberg2012a} have been designed and deployed in the developed world countries. Each of these interventions has demonstrated the potential of Information and Communication Technologies (ICTs) in supporting parents of premature infants while taking advantage of the ubiquity of the mobile phone. Although these interventions significantly improving  parents satisfaction, they require users to have high speed internet connection and accessibility of smart phones which most families in low-income settings can not afford. 

For instance, Baby CareLink provided information to parents using  both a website and videoconferencing system from the NICU \textcite{Gray2000a}. The parents used secured password to access their infants health information. MyPreemie application \textcite{Doron2013} for iPhone and iPad required the parents to buy the application from Apple iTunes store before they could use it. Estrellita \textcite{Hayes2014} includes two interfaces: a mobile application and a web-based clinical interface. The mobile application, allows users to record observations of daily living (ODLs) for the infant and caregiver, share these data with clinical providers, and visualize past recorded ODL data. Through the website, healthcare providers can interact with the caregiver and keep abreast with the infant’s ODLs through a series of simple visualizations and data summaries.

In addition to these interventions requirements not being feasible in low-income settings; due to the high cost of internet and expensive devices \textcite{Mars,Wamala2013}, the researchers did not include parents in the design process. Health practitioners shared the systems requirements with system developers who built the tools. Parents were only included during the evaluation process. According to Balaam et al. \textcite{Balaam2013} designing with new parents especially the mothers bring new challenges for participatory design methods. Parents tend to focus more on their children rather than on the design activities. For instance, D'Ignazio et al. \textcite{Ignazio2016} involved a group of mothers and experts in the design process that focused on improving the design of breast pump. They identified that designing for the postpartum experience is complex and context-sensitive, as it sits at the intersection of numerous legal, political, social and cultural factors. Wardle et al. \textcite{Wardle2018a} mention that mothers availability during the co-design process is limited and Human-Computer Interaction (HCI) researchers should consider appropriate research methods that will allow mothers to fully participate in the process of designing their technologies.
\subsection{Designing Technologies with Parents of Preterm Infants}
Parents of preterm infants start their experience of parenthood in an unfamiliar and intimidating environment of NICU \textcite{Obeidat2009}. These parents are susceptible to stress and conducting research with them require the researchers to consider the sensitivity of the topic. In addition the hierarchical infant care structure in the NICU creates power imbalance which requires careful methodological and ethical consideration before engaging parents in a co-design process. According to Waycott et al. \textcite{Waycott2015}, one of the tensions in HCI research related to sensitive personal issues is the need to balance the generation of credible data with the protection of research participants against potential emotional risks associated with their participation.

For this purpose, researchers have questioned whether HCI researchers are sufficiently equipped to respond to the needs of vulnerable participants \textcite{Vines2013}. Chan et al. \textcite{Chan2017} recommends that researchers should consider (a) what the effect of participation could be for the respondent and (b) how the methods used could affect participation, disclosure rates, and the validity of the information provided. 

From literature, no work has recorded the involvement of premature infants' parents in design process of sociotechnology that could support them while in the NICU. So, how can co-designing with these group of participants from low-income settings affect them? which research methods are appropriate to ensure that their psychological state does not deteriorate?

\section{Research Statement}
The research reported in this thesis seeks to explore the appropriate approach of involving parents of premature infants in the design process of a communication tool that could support them while their infants are hospitalized in the NICU. The NICU staff mostly interact with infants’ mothers to educate them on breastmilk expression and skin-to-skin care commonly known as Kangaroo Mother Care (KMC). They also interact with the few fathers in the NICU to update them on the infants’ health status. To understand why there are few men in the unit, we consulted the unit management, who informed us that most mothers were single parents. For those who were married, we identified that their spouses work during the day thus making them unavailable in the unit. 

This being the case, we opted to work with mothers and not both parents. This ensured that both single mothers and mothers with NICU absentee partners were incorporated in the study. Our participants recruitment decision was also supported by literature  which proves that mothers are the main caregiver of preterm infants in most NICU in low-income settings \textcite{Franck2003, McGrath2013}. Our goal is to explore the appropriate research methods that could be used or how the current research methods can be modified to ensure that mothers are fully engaged in the co-design process of an NICU communication tool. Furthermore, we wanted to investigate whether the intervention designed by mothers is able to meet their communication needs while their infants are hospitalized in the NICU.

Throughout the study, we employed co-design approach engaging mothers and NICU staff in all phases of the design process.In the early stages of this research, we volunteered to work at the NICU and focused on helping nurses to cup feed and clean infants. We used this  opportunity to familiarize with the NICU environment as well as to understand the challenges that both staff and mothers face. We later through field work and literature review sought to identify the best research techniques that could engage both NICU staff and mothers in a constructive participation despite the existing hierarchy in NICU infant care. This study was organized in six phases namely: problem identification, ideation process, ideas specification, interactive prototyping, deployment, handover and evaluation. This study was guided by three research questions, as presented below.
 \begin{enumerate}
     \item How do NICUs and mothers currently use ICTs?
     \begin{enumerate}
     \item Are the current ICTS intervention supporting communication between NICU staff and Mothers?
     \item What are some of the challenges facing the current mode of communication?
      
     \end{enumerate}
     \item  Do existing participatory methods support useful design dialogue with mothers? \begin{enumerate}
     \item If not, what modifications or alternate approaches will empower mothers to actively participate in co-design process?
     \end{enumerate}
     \item What intervention and improvement to current communication modes can be employed to support mothers of preterm infants?
 \end{enumerate}

\section{Research Contribution}
This research makes two major contributions as stated below.
\begin{enumerate}
     \item The first contribution is methodological......
     \item The study also contributes towards the HCI empowerment framework......
      
     \end{enumerate}

\section{Thesis Outline}
The rest of this dissertation is structured as follows:

In \textbf{Chapter 2}, we examine the literature related to provide premature births statistics, challenges that mothers of premature infants face, the communication challenges in the NICU and the current technological interventions used to support parents of premature infants. We uncover the design gap that exist within the design of NICU technologies and share recommendations provided by other researchers conducting research in similar sensitive settings. In addition we discuss the study context providing a rationale for conducting our study in South Africa tertiary hospital.

 \textbf{chapter 3}, provides a thick description of the methods used in this study. We present our reflection on our experiences during this study presenting the methodological and ethical challenges we encounters as we engaged multiple stakeholders in a sensitive research. We examine the literature related to co-design and participatory design and discuss the identified strengths and challenges.The chapter concludes by reflection on the methods used in this research by revealing our findings and perspective as we conducted the research. We present our suggestion and recommendation of methods that were applicable in this context.

 \textbf{chapter 4}, describes in details our experience as we interact with  NICU staff and mothers to identify the communication challenges in the NICU. We further explore their views and perception towards the use of technology in the NICu and share their design ideas. We demonstrate the importance of prioritizing the voices of participants in the design process by discussing the research techniques that worked and those that did not. The chapter concludes by presenting the results of the first and second phase of this study.
 
 \textbf{Chapter 5} presents the activities of prototyping phase. We describe how mutual learning concept was used to disentangle dialogue and discuss the research techniques that were used.The chapter also discusses the therapeutic experience that participants encountered during the session emphasising on the techniques that promoted this experience. This chapter concludes by sharing the proposed designs for the most viable technological intervention.
 
 \textbf{Chapter 6} then presents the deployment of the co-designed NICU communication intervention. We discuss the evaluation process and the approaches used to elicit feedback from the staff and mothers.The chapter concludes with a discussion of the lessons learned after the mothers usage of the intervention.........
 
  \textbf{Chapter 7} presents the reflections by the researcher, based on the findings of all the six phases of this research. In addition, it also discusses how the whole study addressed the research questions by providing insights on design lessons learned from the strengths and weaknesses of the NICU communication intervention. At last it summarises the contributions,limitations of this research and ideas for future work.
  

\section{Chapter Summary}
This chapter has introduced the thesis, describing the communication challenges that  parents face while their preterm infants are hospitalized in the NICU. It presents the current technologies used to support these parents and evidence the design gap and the need of holistic involvement of parents in the design of technological interventions that are meant to support them while their infants are hospitalized in the NICU.It then states the research statement, research question and conclude by giving the thesis outline  for the remaining chapters of this document.
 




