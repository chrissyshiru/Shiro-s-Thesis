% Chapter 5

\chapter{Co-designing with Mothers and NICU Staff} % Chapter 5 chapter title

\label{Chapter5} % For referencing the chapter elsewhere, use \ref{Chapter5} 

%------------------------------------------------------

Phase Two: Idea Generation Phase
Introduction
In this phase we describe the idea generation process followed in our research, sharing experiences and outcomes from the four focus groups we held with different stakeholders. In this phase, we shared findings of phase one with all stakeholders and focused on expounding the new ideas that emerged. In the sections that follow, we articulate both what we learned about co-design in under-researched field and the outcomes of our activities concerning the design of the NICU communication tool. We show how our methodology choice helped us to involve the different stakeholder's voice in the co-design process thus resulting in more refined and appropriate design requirements for the communication tool. We suggest that continuous recruitment of participants and a phased research approach is an essential requirement for sensitive and under-researched Human-Computer Interaction for Development (HCI4D) research.
Methods
Participants Recruitment
This phase built upon the requirements identification phase which involved 15 mothers of premature infants, five doctors and ten nurses. To further discuss the finding of the first phase we chose to include three doctors, four nurses and four mothers who were involved from the onset of the project and who had worked/ or stayed in the unit for the longest period. In addition, we opted to recruit two nurses and two mothers to allow new ideas in the discussion as well as to help us clarify the findings we identified in the first phase. We chose to use focus group and brainstorming sessions to generate design ideas and uncover unexpected areas of innovation in this context. These sessions were held with different groups of stakeholders to overcome the power dynamics challenges that exist in the medical environment. In total there were four focus groups consisting of four doctors, four nurses and four mothers of preterm infants.
The focus groups took place at Groote Schuur hospital which provided a private environment for the participants.  We approached the NICU staff and arranged a convenient time to hold the meetings which happened successfully. However, recruiting mothers whose infants were out of the hospital for at least three months was one of the challenges we faced during this phase. We approached mothers at Mowbray hospital where they attend infants’ routine check-up and they agreed to participate in the focus group. However, when we called them to remind them of the event, some turned down our calls and others had other commitments and could not attend the session. As a result, we had to reschedule meetings twice and on the third attempt, two mothers appeared and two participated through a phone call.
The researcher briefed each group about the meeting objective and consent was sought and secured in writing from each participant before the focus group commenced. We used a voice recorder during the sessions and the respondents were informed that the voice recordings would be destroyed at the completion of the study. The NICU staff were offered a chocolate bar to appreciate them for the time and the mothers were offered 100 rands (about 7 USD) as compensation for lunch and transport.
Idea Generation Process
Understanding the process of idea generation is crucial especially when working with participants with limited design skills. (Hussain, Sanders, & Steinert, 2012), propose that the researcher conducting participatory design in the marginalized area should not only focus on developing a tangible solution but also focus on empowering participants psychologically. This approach raises user ability and confidence to communicate their own ideas and to engage in the design processes. However, (Ertner, Kragelund, & Malmborg, 2010) mentions that participants empowerment through design is not merely associated with a positive result of user participation, but a complex and challenging activity. Common challenges that affect participation are:  language barrier, cultural issues (Winschiers, 2006), low literacy levels (Thomas, 2013), power dynamics (Robert et al., 2015), inequality (Oliveira, 2011) just to mention a few. The researcher should focus on overcoming these challenges to improve communication and collaboration between stakeholders with different abilities.  
Co-design approach was used to engage the NICU staff and mothers of preterm infants in the idea generation process. Findings of phase one were used as the input of this phase to generate detailed requirements specification for the NICU communication intervention. We focused on investigating the appropriate research methods that would involve people of low literacy and limited technical experience in the process of technology design. We held focus group sessions with different stakeholder groups and brainstormed on several ideas that would potentially increase the chances of finding better design ideas. We choose to hold small interdisciplinary groups to overcome the power dynamic that exists between doctors, nurses and mothers of premature infants and promote participant-driven critical thinking processes.
Through these sessions, we identified power role, limited prior exposure to technology, loss of participants, lack of design skills as the major co-design challenges. This led us to incorporate alternative research approaches to fully involve participants in the research and design process.  In the next section, we describe how we enabled participants to find their voices and productively participate in the generation of new design ideas and user requirements of the proposed communication tool.
Focus Group with Doctors
Among the five doctors we interviewed in phase one, we involved three of them in brainstorming and focus group sessions. We presented the finding of phase one and focuses on discussing the topics that doctors had different opinions on. We also introduced the new themes that are unique in this context (that is socio-cultural factors and power role) and asked the participants to talk about their experience as caregivers in the NICU and challenges they face while interacting with mothers. We also asked them to brainstorm on the possible technologies that were suggested in the previous phase. They discussed each intervention suggested giving its pros and cons and later narrowed down the ideas to interventions which were more viable for this context. This discussion led to a prototyping session where the doctors visualized their design ideas in a workflow. In addition, they used scenarios to explain how the suggested communication intervention will enhance communication between staff and the mothers. Throughout this session, the researcher recorded the conversation using a digital recorder and took notes to capture the statements made by the participants during the discussion. 

                                     Figure 1: Focus group with Doctors
Focus Group with Nurses
We included new nurses in this session to validate the data shared by nurses in the first phase as well as to introduce new ideas to enrich our findings. We asked them to share some of the challenges they face as they interact with doctors and mothers in the NICU. We also asked them to share experiences where socio-cultural factors hindered their communication with the mothers. Based on the technologies suggested in phase one we asked them to narrow down the solution and suggest the technologies that were feasible in their context. 
Among the participants, we had a nurse supervisor who had been in the unit for more than seven years. Unfortunately, the junior nurses were not free to share their ideas and we opted to introduce the card-sorting method to allow all nurses to participate in the discussion. All the nurses wrote down their ideas on a sticky note and later they were asked to explain and elaborate on the written ideas. This method helped to engage the junior nurses in the session and at the end, we had a constructive discussion where the nurses listed the solutions that were feasible in this context. 
Focus group with mothers
The goal of this session was to learn about mothers’ experience in the NICU. We focused on understanding how socio-cultural factors affected their interactions with the NICU staff. We also asked them which information they need most and which technology they would prefer to use considering the limited access to advance technology and the high cost of data. The mothers had limited exposure to technology and we opted to use scenario to intrigue their design thinking. This method helped the mothers to collaborate and generate design ideas which we later brainstormed on.
During the session, we had to pause the discussion numerous time as the mothers rushed out to pick calls from their families back at home. In addition, mothers who were on phone call dropped the calls several times when their infants interrupted by crying. This experience resonates with Balaam et al [3] findings which prove the importance of developing design methods that are easily paused and re-started when working with mothers with young kids to accommodate the numerous need of their young children.
Joint focus group
This was a brief session that focused on discussing the findings of all the focus groups to ensure that the stakeholders agreed with the final findings of this phase. We shared the findings of each group and allowed the stakeholders to discuss the findings and agree on the most viable approach to solving the communication challenge between the mothers and the staff.

                                          Figure 2: Joint Focus group

Data Analysis
Data analysis began in the field; during the session, we wrote field notes documenting participants comments and reactions. The focus group sessions data were transcribed and coded and categorized into themes using Nvivo software. Affinity mapping process was used to organize these categories into groups based on their relationship. The credibility of these findings was enhanced through triangulation, which involved the analysis of our field notes, digital photographs and focus group transcripts.
Findings
In the next section, we present the findings from our studies; in particular, we reveal how sociocultural factors and power dynamics affect communication in the NICU, the challenges of the current communication systems and perception of stakeholders towards the use of technology to enhance communication in the NICU.
Factors Affecting communication in the NICU
Sociocultural factors
The participants mentioned several cultural factors that acted as barriers to communication in the NICU. All mothers confirmed they follow religious and cultural practices which most NICU staff do not consider while interacting with them. As a result, mothers feel their culturally influenced viewpoints are neglected and not included in the decision making of their infants’ care. The comment below was made during the session:
“They do not incorporate our cultural practice in the treatment of the baby. We are afraid to raise them because only people of our culture can understand them”
The NICU staff were in agreement with this accusation from mothers. Two doctors mentioned they did not understand most of the South African culture thus sometimes they are unable to provide compassionate care. However, in cases where mothers are blamed by their partners/family for not carrying the pregnancy to term, they do intervene and offer support. For example, one doctor said:
“Some mothers are blamed by their family for not giving birth to a full-term infant. They relate this predicament to not performing some rituals while they were pregnant or for giving birth out of wedlock. We provide counseling to such mothers and involve social workers to follow up with them back at home”
On the other hand, participants agreed that some cultural practices go against the unit rules and regulations. The space in the NICU is small and as a result, only parents and grandparents of the infants can visit the unit from 8 a.m. to 7 p.m. However, some mothers invite spiritual leaders during late hours and they are restricted from entering the unit. In addition, some spiritual practices are not allowed in the unit and this contributes much to the miscommunication and tension between mothers and staff. For instance, these two comments were heard during the session:
“I invited an imam to pray for my baby, but the security guard could not allow him in the unit and I had to get intervention from the nurses. This is so stressful that we are not allowed to conduct our religious beliefs in the unit”
“I wanted the pastor to hold my baby and apply holy water on her forehead, but the nurse said we could not remove the baby from the incubator. This was devastating, and I felt I could not make any decision with regards to my baby”
In response to these comments, the staff mentioned that they put all these restrictions to ensure that the infants’ health improve. During the joint focus group, all participants agreed that mothers should be educated on the unit’s policies to ensure there is no conflict between staff and mothers. The unit should engage with new mothers in the unit and take them through the unit rules and later issue a brochure with all the unit rules and policy which the mothers should sign to indicate they will adhere to them. In addition, all participants agreed that the final design should consider the different cultural practices and languages in South Africa to support the needs of the specific population.
 Power Dynamics in the Unit
Mothers perceived a power dynamic between themselves and the NICU staff, which is due, in part, to a complex interplay between the language barrier and different roles in the unit. In NICU settings, the balance of power favors the doctors and nurses because they spend most time with the infants. As Foucault puts it, non-sharing information between health professionals and patients is an appropriate arrangement for the exercise of coercive power since it hinders informed consent and can lead patients in the direction of an act of agreement with a perspective other than their own (Baptista, 2017). In this light, coercive power applies to health institutions in different ways by which health practitioners have dominion over patients and their family member who take care of them.

Several studies underscore the fact that the hospital system is embedded in a hierarchical structure where the voice of the health care provider as an expert is often given more importance than the patient or patient’s family members (Bristowe & Harris, 2014) (Nimmon & Stenfors-Hayes, 2016). In the NICU context, staff-mother relations are complex and although these relationships may appear harmonious on the exterior, sometimes mothers may be harboring negative feelings on the inside (Obeidat, Bond, & Callister, 2009) (Lupton & Fenwick, 2001). 

Through our interaction with different groups of stakeholders, we identified that there exists power inequality in the unit which hinders communication between the caregiver's team and, consequently, the relationship between them ant the mothers. Most mothers fear and have negative feelings toward some staff due to different reasons. They fear to approach doctors and they keep their thoughts and feelings hidden despite experiencing a whole host of emotions. The doctors mentioned language barrier as the main hindrance to their interaction with mothers. To overcome this barrier, doctors work closely with nurses who interact with mothers in their vernacular language. One doctor reported:
“Sometimes we ask the nurses to interpret the medical information to the mothers on our behalf. This help the mothers understand that we are there to help them and we are interested in supporting them “
However, this approach is not effective because sometimes the nurses are not able to interpret the medical terms in the vernacular language. In addition, the doctors are not able to understand whether the information relayed is correct and most time they do not receive any feedback from the mothers. Moreover, we identified some mothers fear some nurses who scold or are rude towards them. Expanding on this, one mother explained:
“The woman who sat next to me did not like the way the nurse handled her baby but she couldn't confront her. She told her husband who wanted to report the nurse to the management, but she stopped him because she feared the nurse will mistreat her baby. Unfortunately, the baby died that night and the lady felt guilty for not fighting for him”
All mothers recalled several situations where they were frustrated by the NICU staff’s actions but were reluctant to confront them. For example, one mother did not fully understand why the nurses were nonchalant about monitor alarms. She had to learn for herself that many of the beeps and buzzers were false alarms, but only after a few frightening experiences. 
Another situation evidenced was that the doctors and nurses work closely to provide services in an efficient manner. However, doctors and nurses occupy different professions and jurisdictions, which is defined by professional boundaries. The nurses receive orders from the doctors and sometimes they do not respond to mothers’ queries until they confirm the medical information with the doctors. For instance, one nurse said:
“Communication between nurses and doctors is good. We receive instructions from them on medication, feeding and hospital or section transfer. Sometimes when we can’t explain the medical condition to a mother we ask the doctors to do it. However, they are very busy”

Another interface where power differences played out between doctors and nurses was during the ward rounds. The conversations during ward round are led by doctors and nurses only respond when detailed information about infant feeding or medication intake is required. The technical and socioeconomic inequalities between doctors and nurses hinder their communication and their relationship which consequently, interferes with the relationship with the mothers.
In the next section, we elaborate how both sociocultural factors and power dynamics in the NICU affected the views of participants in the selection of the appropriate communication intervention in the NICU context. 
Participants Views towards Use of Technology
The overall expectations of all stakeholders towards the use of technology in enhancing communication in the NICU were positive, although they expressed concerns in some of the approaches suggested in the first phase. Their decisions and perceptions were influenced by the sociocultural factors and hierarchy of power among participants in the NICU.
Table 1 summaries the pros and cons of the technologies previously listed as the viable means of disseminating information.
Table 1: Pros and cons of suggested technologies
Information Needed
Suggested Technology
Pros
Cons
Breastfeeding information
Digital video
Digital video will be projected on the screen in the expressive room

Most mothers phone have limited capacity to hold the video 

Text messages
An easy way of disseminating information because mothers with both featured and smartphone can access the message
Mother can retrieve the message and read it again


It is not ethical to share patients’ data via text messages due to confidentiality issues
Mothers provide incorrect phone numbers we won’t be able to reach them through this mode


Interactive website
Mothers with smartphones will be able to access this information 


The high cost of internet mothers won’t be able to access the information 

Neonatal Information
Text messages
An easy way of disseminating information because mothers with both featured and smartphone can access the message

It is not ethical to share patients’ data via text messages due to confidentiality issues
It is cumbersome to share neonatal information because infants condition changes quite often
Most mothers give incorrect phone numbers and this may lead to a breach of confidentiality

Digital video
Common conditions which are related to prematurity can be projected on the big screen in the NICU.


Some mothers share a phone in their household thus this may lead to a breach of confidentiality 
Mothers phone have low capacity and can’t store video with large data

Use of toll number in location health facility
An easy way for other to call NICU and request for infants’ information
Will support mothers who can afford transport and phone call cost

Expensive approach because phones should be installed in various health care facilities and personnel employed to operate them

Hospital transfer Information
Use of text messages
An easy way of disseminating information because mothers with both featured and smartphone can access the message

The unit transfer message might give the mothers false hope since infants’ health change drastically
Mothers provide wrong phone numbers. It is hard to reach them

During the joint focus group, we shared the suggestions of each group and brainstorm on each idea. All stakeholder groups mentioned that we should focus on sharing breastfeeding information and common medical terms used in the NICU. However, they agreed that the generated digital video should not show the mothers and infants but rather they should include only text and voice. They agreed that neonatal and infants transfer information was confidential and technology such as mobile phones could not be used to disseminate this information. Patient confidentiality is enshrined in South Africa National law and it is an offense to disclose patients’ information without their consent, except in certain circumstances (Republic of South Africa, 2004). This right to confidentiality means more than simply refraining from sharing information but one is also responsible for ensuring that all records containing patient information are kept securely.  In addition, the mothers mentioned that neonatal and transfer information would cause the mother to be anxious, especially if she is not in the unit. For instance, one mother said:
“I would not like to receive my child’s health condition via SMS, it is traumatizing. There is a time my child was transferred between ICU and high care several times. I was scared every time I received the call from the hospital. At some point, I asked the doctor to keep him at ICU coz he was ok while there. Every time he left ICU he got sick”

To ensure that mothers understand the common medical terms used in the NICU, all stakeholder agreed that educative video can be recorded and shown on the screens in the NICU. This would help the mothers understand doctors’ conversation during the ward round.

In relation to breastfeeding information, the stakeholder mentioned that we should focus on supporting mothers who are no able to visit the unit regularly. The doctors suggested the use of technology to enhance MOM Project (a project that motivates mothers to express and deliver their breastmilk to local health care facility which is later collected by scooter drivers and delivered to GSH NICU kitchen) and consequently use this technology to encourage mothers to express breast milk. We presented the idea to the nurses and mothers and asked them to provide their views.
 This approach was somehow similar to nurses’ idea who suggested that we should have an application that allows the unit to record the breastmilk delivered by the mothers at the unit and upon delivery, the system sends scheduled messages to the mothers based on the age of the infant. Unfortunately, the mothers did not have much knowledge about the MOM Project and they had suggested the use of automated text messages to educate them on educated them on the feeding schedule at the unit and the amount of milk fed to their infants. This prompted the nurses to describe the Mom Project initiative to the mothers in details and to help them understand the doctors’ suggestion, we presented the workflow sketched by the doctors. Below is the workflow designed by the doctors

Figure 1: Workflow Suggested by doctors

The nurses and mothers embraced the idea and to complement it, all participants agreed that we should project digital videos in the NICU expressive room screen to ensure continuous education of the importance of feeding premature infants with breast milk. The mothers claimed they get bored in the NICU and watching such information would educate as well and keep them engaged.

Discussion and Design Implication

We present lessons and implications for designing with and for mothers of premature infants in developing world contexts.  We focused on understanding the best approach of co-designing with vulnerable participants in low-income contexts.

Participants Empowerment
We learned that participants empowerment triggers a higher sense of engagement and enables people with different backgrounds and skills to cooperate creatively looking for a solution that meets all their needs. This results in mutual learning between all members of the design team and foster interaction between stakeholders. (Wilson et al., 2015) argue that co-design techniques encourage designers and participants to work together in the creation of design solutions, but often make assumptions about the ways in which participants will be able to communicate. This leads to the unwitting exclusion of shy or non-expressive people from the design of technologies. To overcome this challenge, we chose to have separate focus group sessions with the three groups of stakeholders to ensure participants were comfortable around their peers before meeting in a joint focus group.

Having separate sessions with different group of participants did not only encourage participants to develop ideas for the communication tool but also empowered them psychologically by raising their ability and confidence to take part in developing solutions that meet their needs. This was evident when the doctors took the initiative of sketching the workflow of the desired approach and role-played how the intervention will be used in the real environment. This observation resonates with (Marsden, Maunder, & Parker, 2008) findings which prove that when people with limited exposure to technology are empowered they are capable of shaping technology solutions to meet their own needs. In addition, the stakeholders had a constructive discussion during the joint focus group session where they brainstormed on the suggested ideas and narrowed them to the most viable approach. We allowed them to lead the discussions keeping in mind that they were experts in this context and as the researchers were involved in the process to learn from them. According to (Zimmerman, 1995) participants empowerment contributes to building local human capacity and enable people to undertake and lead their own design projects.

We argue, therefore, that the quality of the ideas generated by the participants during the design process would have been compromised if we did not empower them to own and lead the brainstorming session. Our objective in this study was to follow a design process that would take the participants’ limited exposure to technology and design skill into consideration while recognizing them as the experts in matters relating to NICU environment and communication challenges. We identified that when stakeholders’ creativity and voices are curbed, we the designers end up making major design decisions yet our knowledge of the NICU communication challenges is limited. In addition, the separate brainstorming session empowered the participants to be comfortable among their peers thus empowering them to confidently voice their suggestion in the joint focus group.  

Methodological Consideration in Sensitive and Under-Researched Studies
Our approach was to empower participants to own this project as we focus on understanding the methodological approaches that are appropriate while conducting research in a sensitive context.  Qualitative research is more suited to the study of sensitive topics as it does not assume prior knowledge of people’s experiences (Lee, 1993). Instead, it allows people to develop and express their own reality. We identified that there was a hierarchy of power among our participants and we were curious to identify the research methods that would encourage engagement in the design activities. While conducting sensitive research, we argue it is important to investigate the methodological issues and to examine them from the perspective of both researchers and participants.
Focus groups have increasingly been employed in health-related inquiries but considerations about the sensitive aspects of such research are not often seen in research reports or discussion on ways of conducting sensitive research (Oliveira, 2011). In this study, we conducted four focus group and each session involved a mixture of different research methods to ensure we acquire useful feedback from participants. 
Firstly, in the doctors’ focus group, we identified that all participants ideas built on each other’s comments. The discussion was fully controlled by the participants and the researcher only chipped in while posing clarification questions. This prompted the participants to visualize their ideas and even involved role-playing to demonstrate how their ideas should be implemented. This approach helped both the researcher and participants to identify the gaps within the suggested workflow and this encouraged further brainstorming which resulted in the idea of using both the milk delivery system and digital video in the NICU unit.
Secondly, during the nurses’ focus group session, we identified that the junior nurses were not vocal in the presence of their supervisors. We introduced the card-sorting method to motivate the nurses to write down their ideas and later elaborate on them. This encouraged everyone to participate and new unique ideas raised which resulted in a rich and productive discussion. We identified that researcher’s relationship with participants is essential while conducting sensitive research. The researcher can easily learn participant body language when they are uncomfortable and interject with a different topic or activity that will encourage participants to fully participate in the discussion. (Oliveira, 2011) mentioned that sensitive research should be flexible enough to be totally or partially changed to accommodate new topics and activities which can mitigate risk and under participation which are common in such studies.
Thirdly, we incorporated scenario during mothers’ focus group session to help them understand how technology can be used to enhance communication in the NICU. This approach was prompted by a mother’s statement who said:
“We don't have details about this project. I would prefer if you helped us understand how it works.”
Most of these women have limited exposure to technology and design skills. To adhere to (Maunder, Marsden, Gruijters, & Blake, 2007) approach, we had to localize the focus group session to enable the mothers to envisage the use of technology in their contexts since they had no experience against which to judge a technology. This helped the mothers to grasp abstract design concepts and understand how technology might aid them to interact with the NICU staff. We argue that it is important for researchers to introduce new techniques or tools that can expose participants to design concepts early in the design process, thus avoiding costly changes late in the design process.
Lastly, researchers should evaluate the benefits and drawbacks of having joint focus group sessions with users and other stakeholders in each specific case. It is important that users’ voices are heard and that users are included in the design process, but it should not uncritically be assumed that gathering all types of participants in one meeting or design activity is always the ultimate goal. 
Cultural Consideration in Sensitive Research
In this phase, we identified that having a profound knowledge of the local culture and society is essential to motivate people to participate and to organize design activities in a culturally appropriate way. During the interaction with participants, we delve into common cultural practices that affected communication in the unit. We were able to get cultural views from all stakeholders and focused on understanding how they should be embraced by the NICU to ensure they mothers feel respected.  Culture strongly influences people’s values, expectations, behavior, and even perceptions and cognitive reasoning (Miller, 2014). Researchers conducting research in sensitive and under-researched context should set aside enough time to understand the local culture and use this understanding when engaging with participants. The best way to get cultural insight is to spend time with different stakeholders, and not only the intended end users. Being flexible and adapting participatory design methods to the local situation and cultural context is necessary.
In addition, the inclusion of culture-specific design specifications and the creation of a specialized design for a target audience is an essential consideration for designers and HCI researchers in marginalized regions. In meeting the needs of the target participants, design specifications must be true representations of their needs. One way of authenticating design is through the use of ethnographic research (Young, 2008). Ethnographic research seeks to describe people’s ways of life or culture. Moreover
Although there is influential literature about culture in technology, recent literature claims that cultural diversity has become a new challenge for HCI (Cardoso et al., 2015). Therefore, researchers should approach culture as an important subject that is transversal to some HCI challenges. all design project should be based on a strong understanding of the history, culture, and society of where the product will be used.
Conclusions and Further Research Work
We engaged in separate focus groups with different stakeholders and later held a joint focus group with the purpose of generating requirement specification of NICU communication tool. We empowered the participants to run the discussions and through this, they were able to voice their needs and provide suggestions of the most viable approach of tackling the communication challenge between NICU staff and mothers of premature infants. Our findings confirm that it is essential for the researcher to blend or localize design methods to ensure that all participants engage in the design process. In addition, to avoid power dynamics among participants, we advocate a separate meeting of stakeholders to ensure they are comfortable with their peers before meeting the rest of the design team. This approach allows participants to engage in productive talks about how technologies could be used in their work. Finally, we identified the importance of incorporating participants culture in the design of technologies. Culture itself is a challenge for HCI and it requires researchers to jointly redefine traditional HCI theories, methods and practices.
In the next phase of this study, we plan to use cultural probes and other artifacts to help our participants sketch and design the technology that they will use to enhance communication in the NICU particularly focusing on how best we can educate mothers on breastfeeding information and common medical terms that are used in the unit. Initially, we will use the low-fidelity prototype to present their designs and interact with them. Later we will design interactive high-fidelity prototype which we will evaluate with participants before we commence with the actual development. Through this process, we will focus on understanding which research methods work well with our participants and ensure that the final prototype represents the stakeholder's needs.