% Chapter 2

\chapter{Background and Literature Review} % Main chapter title

\label{Chapter2} % For referencing the chapter elsewhere, use \ref{Chapter2} 

%--------0. Preamble-------------------
%--------------------------------------
\section{Premature Birth}
Premature birth is defined as birth before 37 weeks of gestation \citep{WHO2015, Blencowe2012}.  According to Lancet report on preterm births, an estimate of 15 million births or 11 percent of all births worldwide occur prematurely \citep{Blencowe2012}. Globally, preterm birth is a major cause of neonatal deaths (death under 28 days of age) \citep{Blencowe2012, Morisaki2014}. Although great progress has been made over the last decade to improve care of preterm infants, the reduction of neonatal mortality has been much slower accounting for 45 percent of the global neonatal deaths \citep{TheLancet2008}. The majority of these deaths are in developing countries, and in particular in Africa  and South Asia regions \citep{Koenraads2017}. Additional to its contribution to mortality, preterm birth has lifelong effects on infants neurodevelopmental functioning such as increased risk of cerebral palsy, impaired learning and visual disorders, and an increased risk of chronic disease in adulthood \citep{TheLancet2008}. 
\par
The  health problems associated with preterm birth is accompanied by high cost in terms of neonatal intensive care and lifelong physical, neurological and educational disability needs \citep{Behrman2007}. These costs impose a considerable burden on finite health care resources especially in low-income countries. The emotional cost is also high, with many families experiencing the sudden loss of a preterm baby or a stressful hospital stays, sometimes for months \citep{Blencowe2012}.
\subsection{Stress Related to Premature Birth}
Premature birth and infant hospitalization are stressful events for parents especially the mothers who are supposed to  nurse the hospitalized infant \citep{McGrath2013}. The experiences of mothers during their infants’ NICU hospitalizations are well-documented in the literature \citep{Steyn2017, Franck2003, Turner2015}. Giving birth unexpectedly and subsequent separation from their new  born is worrying to parents. Moreover, they are overwhelmed with grief and fear during this period making them feel helpless and uncertain of their infants' health outcome \citep{Blencowe2012}. They lack critical skills required to partake in the care of their sick infants thus leaving them with no option, but to rely on the instructions provided by the NICU staff.

To partake in the care of their hospitalized infants, parents, especially mothers are forced to make major life adjustment. They have to travel to the hospital regularly as well as balance other aspects of family life which creates complexities in their daily life \citep{Heidari2015a}. Even worse, is the possibility that these infants may need  to  spend  a  prolonged  period  of  time  in  the  NICU. Furthermore, parents are overwhelmed by the technological environment with unfamiliar equipment, displays, blinking lights, and noise which make them feel uncertain and insecure about their infants' life outside that environment \citep{Ionio2016}. 

Previous studies prove that most parents in the NICU have shattered confidence and they avoid touching their infants through fear of giving them a harmful infection \citep{Arnold2013, Ionio2016, Heidari2015a}. As a result, they turn into mere spectator as NICU staff perform procedures on their infants \citep{Araujo2010}. This exclude parents from full involvement in infant care making them feel guilty for not knowing how to take care of their own infants. Hence, this situation causes parents to develop negative feelings toward the health care profession team and they consequently distance themselves from NICU staff thus limiting the much needed communication. 

 NICU staff, are best placed to provide support to parents because they come into daily contact with them \citep{Orzalesi2011, Enlow2017, Mok2006}. However, NICU staff are overwhelmed with heavy duties in the unit which make them prioritize on infants' well being thus neglecting parents' psychological needs \citep{Kadivar2017}. To reduce stress related to premature birth, parents desire more support from NICU staff, particularly in the area of supportive communication and the giving of infant health information \citep{Enke2017}. Addressing parents' psychological needs is essential to reduce stress related to premature births. This can be achieved through interventions that focus on enhancing communication  between NICU staff and the parents. However, NICU communication is faced with numerous challenges thus making  NICU staff-parents interaction not to achieve it's full potential. In the next section we discuss NICU communication challenges.  

\section{Communication Challenges in the NICU}
Previous literature has shown that communication between NICU staff and parents of hospitalized infants if faced with numerous challenges \citep{HadianShirazi2015, Campos2017, Enke2017, Fowlie2007}. Reason being that many of the ethical and medical issues that are encountered routinely in the NICU are highly complex and have to be communicated to parents who are under extreme pressure in an intimidating environment \citep{Enke2017}. NICU staff are expected to give appropriate and timely information to parents and communicating empathetically \citep{Campos2017}. 

According to Mercer and Reynolds \citep{Mercer2002}, empathetic communication in clinical settings comprises four components: 1. the ability to subjectively experience another’s feelings (emotive), 2. an altruistic force that motivates empathetic practice (moral), 3. an understanding of the other person’s perspective (cognitive) and 4. the ability to act in a helpful way that is based on a validated understanding (behavioural). However, Campos et al. \citep{Campos2017} and Weis et al. \citep{Weis2014} highlight that NICU staff are not well equipped to share sensitive information with the already stressed parents.  

Most NICU especially in developing world context, still follow traditional NICU system where infants are under the supervision of NICU staff only, without much parental involvement \citep{Sankar2017}. This creates a hierarchical care structure which hinders effective communication between the parents and NICU staff \citep{Jones2007a, Brock2015, Kowalski2006, Bramwell2005}.The parents are updated about their infants' mostly during ward rounds and counselling sessions. Miscommunication, inadequate explanations of medical terms and conflicting information from the NICU staffs have been reported as some of the communication barriers in the NICU \citep{Wigert2013, Coats2018, Russell2014}.In addition, the unfamiliar technical NICU environment \citep{Heidari2017} and cultural factors \citep{Rostami2015, Ramezani2014} have been reported to be communication barriers in the NICU. In the next section we describe in detail how these factors affect NICU communication.

\subsection{NICU Hierarchical Structure}
Literature shows that most health environment have hierarchical structure which creates a situation of power imbalance between health personnel and patients/ care givers \citep{Henderson2003, McDonald2012a, Molina2018, Tang2013}. Power can be described as the relationships between two or more entities where one entities’ behavior may affect the other \citep{Bristowe2014}. Power imbalance often emerges due to difference in social, cultural and professional between health personnel and the patients/ caregivers \citep{Tang2013, Rothmann2016}. In the hierarchy of health professions, doctors have traditionally defended their professional autonomy and independence and professional status in their relationships with other health care workers. Moreover, health personnel do not share their knowledge and decision-making role with patients/ caregiver \citep{Henderson2003}. Instead, they tend to control patients/caregivers input thus hindering their contribution in their own health care. 

Within sociology, the work of Michel Foucault has supplied an understanding of medical profession function in clinical settings \citep{Powers2003, Molina2018, Bristowe2014}. Foucault empirical analyses of power has been particularly useful in locating the historical functions of the clinic as a site of bio- power: a concept that describes the impact of interprofessional relationships of patient autonomy \citep{Bristowe2014, Foucault1982, Ameen2017}. This autonomy call attention to the notion that patients due to their lack of medical knowledge are placed in position of vulnerable supplicant when they seek consultation from health personnel. The health personnel exert their power on the patient through different manipulation mechanism and patients have little opportunity to challenge their decision \citep{Molina2018, Powers2003}.

In NICU context, literature shows that the care of hospitalized infants is predominantly done by the staff, limiting the parental role to merely instruction receivers \citep{Jones2007a,Obeidat2009, Heidari2015a}. In most cases, parents; who are the primary caregiver of the admitted infant find it difficult to  express their views and make decision over infant's health in the midst of the health staff. The power inequality in infant's health decision making hinders interaction between NICU staff \citep{Jones2007a, VanMcCrary2014}.  Obeidat et al. \citep{Obeidat2009}, Wigert et al \citep{Wigert2014b} and Ionio et al. \citep{Ionio2016} studies delineate that NICU parents feel powerless, hopeless and alienated within the NICU environment mainly because they lack the information required to involve them in the decision making of their infants' health.  

In addition power inequality exists between doctors and nurses. Wigert et al \citep{Wigert2012} identified that doctors sometime do not directly share information with parents. Instead, they relay the information through the nurses. This limits parents interactions with the doctors who hold vital medical diagnosis of the hospitalized infants. Desai et al. \citep{Desai2017} also identified that nurse-doctor hierarchical relationship attributes to poor teamwork and communication which eventually contributes to reporting errors. Doctors possess greater power in decision-making which causes them to have lesser interest in collaborating with the nurses.These lead to ambiguous communication between doctors and nurses which often lead to  unpleasant behaviours among the NICU staff team. 

A study by Rosenstein \citep{Rosenstein2002} on the perception of NICU staff team towards their working relationship highlighted that nurses often failed to gather all the relevant information before calling doctors to check on the infants. This unclear communication caused the doctors to raise their voices, which significantly affected nurses' attitude towards patient care. Such working relationships have caused nurses to leave the profession, making retention and recruitment of nurses increasingly difficult \citep{Chiswick1987} .This explains the staff shortage in NICU which is critical to effective and efficient communication to parents.Chiswick \citep{Chiswick1987} suggest that it is important to integrate nurses in the medical care team to ensure that they have accurate and adequate information to share with the parents.

\subsection{Inadequate Communication}
Poor and inadequate communication results in patients dissatisfaction, increased complaints and litigation. This occurs when patients/ caregivers lack adequate medical information to make decisions of their health care. In NICU context, the ethical and medical issues that are encountered routinely are highly complex and need to be communicated to stressed parents in an effective manner \citep{Enke2017}. However, several studies have reported miscommunication, inadequate explanation of medical terms and conflicting information as some of the barriers that hinder parental education and parents’ ability to actively partake in infants' health care decision-making \citep{Musabirema2015, Fleck2016, Spiridonov2017, Weis2015 }. 

For instance, in Wigert et al. study \citep{Wigert2013}, many parents were dissatisfied with the information they received from the NICU staff. They mentioned that they did not receive emotional support and adequate information to involve them in the decision making of their infants health. Mok and Leung \citep{Mok2006} identified ineffective staff communication as a source of parental stress.  Parents want to obtain comprehensive, honest and clear information in order to have a clearer understanding of their infants' condition. Underlining the importance of NICU communication, Makworo et al \citep{D2016} reported that lack of space, language barrier, staffing and time limitation was a challenge that made parents not to receive individualized care. Due to lack of essential facilities in public hospital, most parents opt to leave the infants in the NICU thus eliminating them from the infants' care team.

With the health of hospitalized infants at stake, learning to communicate effectively and efficiently with all members of the patient-care team is critical. Shields \citep{Shields2015} recommends a holistic care where parents are equipped with relevant information to promote their partnership with NICU staff in the care of the preterm infants. However, implementing this patient-care team is challenging especially in developing countries because patient to nurse ratios are so high. To make it work, managers of health services should amend the hospital policies to ensure that all those concerned with care of children are equipped with relevant information.
  
\subsection{NICU Technical Environment}
Giving birth to a premature infant that requires admission to the Neonatal Intensive Care Unit (NICU) creates additional layers of responsibility for parents who are already facing a major life adjustment \citep{Barkin2010}. NICU environment is highly technological with unfamiliar equipment, with displays, blinking lights and noise \citep{Heidari2017}. This environment is scary and confusing for most parents. They are most often separated from their infants immediately after birth and during the hospital stay, the infants are cared for in a carefully controlled environment with monitor wires or feeding tubes or oxygen attached to them. NICU is noisy with equipment bubbling, beeping of alarms, crying infants and NICU staff consulting \citep{Rand2014}. 

The NICU environment significantly impacts parents \citep{Williams2018} They face challenges such as transportation to and from the NICU and balancing other aspects of family amidst caring for the hospitalized infant \citep{Mburu2018a}. In the NICU they sit next to the incubator and watch the infant struggling to live. They lack information to help them understand the role of NICU equipment. They expect to get this information from the NICU staff who work round the clock to stabilize infants health thus neglecting to orient parents to the NICU environment.For instance in Kim et al. \citep{Kim2015}  study, parents mentioned that they received little information to support them in infant care. Parents need to know what the machines, wires and alarms do and mean; the rules for touching/holding their infants. If this information is not provided, parents might fear touching their infants lest they disconnect the wires attached to the infant \citep{Arnold2013, KellyOBrien2014}.

Consequently, this interrupts and delays parent-infant bonding and attachment \citep{Heidari2017}. As a result this influences the quality of care offered by parents of the hospitalized infants. Therefore, there is need for supporting parents cope with the NICU environment to enable them to fully participate in the care of the infants. In the next section we describe the environment of the NICU we will be conducting this research.

\section {Groote Schuur Hospital Context}
Groote Schuur Hospital (GSH) is a tertiary, government funded, teaching hospital in the city of Cape Town, South Africa \citep{WesternCapeGovernment2014}. The hospital provides tertiary level neonatal intensive care, obstetric and antenatal services to women with pregnancy complications from the West Metropole of Cape Town. The 75-bed capacity neonatal unit admits approximately 2000 infants annually, majority of which are preterm infants. Most parents of these infants live in informal housing settlements on the periphery of the city where overcrowding, unemployment and poverty are rife \citep{Thompson1993}. 

The NICU is under-staffed. Doctors and nurses work for long hours to ensure the health of the infants stabilizes. parents rely on brief interactions with the NICU staff to understand their infants' health status. Despite the presence of parents in the NICU, some have little information about their infants' medical conditions. This exacerbates their level of stress and they are in dire need of emotional support from the staff. 

In addition, the unit space is small and their in no privacy in infant care. The unit has limited lodging space for mothers who have to travel over 100 kilometres to the hospital \citep{Kapembwa2017}. Even so, parents who live in the peripheral suburbs of Cape Town struggle to raise transportation cost required to travel to and from the hospital on a daily basis. This limits the number of visits they make to the hospital thus prompting them to use other communication channel to enquire for their infants health status.

To mitigate the NICU communication gap, the unit uses communication channel such as phone calls and text messages to enhance NICU communication. In addition they make use of government social workers who visit the parents at their homes to share infants health status. The unit also has limited number of counsellors who provide emotional support to parents in the unit.
However, these interventions are not efficient in this context.

We conducted literature review to analyse other interventions used to address NICU communication challenge. We identified that scale-up of technology and cost-effective interventions such as family-based care, peer support in the NICU could involve parents in the care and decision-making of infant care which may restore their parental role \citep{TePas2017, Ramezani2014, Mendelson2017}.  In the next section we discuss these interventions  and analyse issues that affect them. 

\section{Programs Used to Enhance NICU Communication}
To complement staff-parents face-to-face communication, researchers have explored variety of approaches for equipping parents with information. This includes the baby diary, that allowed doctors to update on infant's progress and parents write in memories or notes as well as questions or concerns for staff to address during face to face communication \citep{VandeVijver2015}. However due to staff shortage, frequently change of staff shift, language and cultural barriers hindered the regularity of staff documentation and parents participation. In the next section we discuss some of the common programs used to support NICU parents.

\subsection{Family-Centered Care in NICU}
Family-Centered Care (FCC) is a holistic care that promotes partnership between parents and health care professional in the care of preterm infants in the NICU \citep{Al-Motlaq2017,Shields2015,Ramezani2014}. This approach, as a team-oriented and multi-disciplinary one, involves families in breastfeeding, kangaroo care, care planning, and limitless presence beside their neonates \citep{Ramezani2014,Al-Motlaq2017}. In addition, it enables the family members to take care of their neonates with less expenses and optimal quality \citep{Ramezani2014}. Studies have identified 11 dimensions of care as important to parents whose infants receive neonatal intensive care: assurance, caring, communication, consistent information, education, environment, follow-up care, pain management, participation, proximity, and support \citep{Ranchod2004}.

However, family-centered care is considered a multidimensional and complex concept where different social, cultural, economic and behavioural factors influence its application in the NICU \citep{Ramezani2014}. Coats et al. \citep{Coats2018} suggest that FCC is a wonderful ideal but difficult to implement effectively because of the human factors that influence the relationships between NICU staffs and parents. As a result, most parents feel their need for communication, is not always met by the NICU staff and staff may not be aware of communication problems in the same way as parents \citep{Weis2014, Coats2018}. Moreover, little is known about the satisfaction level of parents of neonates requiring intensive care in low-income settings \citep{Russell2014, Oulton2011, D2016}.

\subsection{Creating Opportunities for Parent Empowerment (COPE) intervention}
Creating Opportunities for Parent Empowerment (COPE) intervention was designed to strengthen parents’ knowledge and beliefs about their preterm infants and their own parenting role and remove barriers that would inhibit them from participating in their infants’ care and interacting with them in a developmentally sensitive manner  \citep{Mendelson2017, Melnyk2006}. 

COPE program is a 4-phase educational-behavioral intervention program. Each phase provides parents with information on (1) the appearance and behavioral characteristics of premature infants (infant-behavior information) and how parents can participate in their infants’ care, meet their infants’ needs, enhance quality of inter- action with their infant, and facilitate their infant’s development and (2) activities that assist parents in implementing the experimental information \citep{Melnyk2004}.

According to Mendelson et al., \citep{Mendelson2017} they identified that for parents utilizing COPE intervention, had reduced maternal anxiety and depressive symptoms. In addition the intervention proved that parents had built stronger confidence in their infants' care. As a result, NICU staff perceived that the parents were ready and able to take their infants home thus expediting the discharge of hospitalized infants. However, most parents refused to participate because they were reluctant to make additional commitments \citep{Mendelson2017}. Moreover parents dropped out from the program due to infant deaths, infants discharges or transfers thus it was difficult to evaluate the  effectiveness of the intervention.

\subsection{Peer-to-peer Support Program}
 Peer-to-peer support is a well-established modality for improving outcomes in people with a wide range of risk factors and diagnoses \citep{Humphreys2004, Dixon2014}. In a NICU setting, peer-to-peer support program has been used to support parents. The 'veteran' parents  act as parent-staff mediators by using their experience in the NICU to support current NICU parents \citep{Sorkin2016}. Literature has shown that there are many benefits of peer support for NICU parents: They become more optimistic, confident and accepting of their situation, and they develop better problem-solving capabilities and spend more time with their infants, with greater feelings of empowerment \citep{Hall2015, Sorkin2016}. Rates of parental depression and anxiety are lower in NICU parents who have had a peer match \citep{Sorkin2016, Hall2016}. However, the program faced numerous challenges such as staff being skeptical that parents can meet other parents' needs , lack of space and limited time and resources to support volunteers \citep{Hall2015c}.

\section{Technology Used to Enhance NICU Communication} 
Several technologies have been tested to improve communications and promote interactions between NICU staff and parents \citep{Hayes2014, Doron2013, Gray2000a, Globus2016, Mahamood2011, Weems2016}. Although initial findings are positive, research in this area is quite limited and the reviewed studies had several limitations. 

To understand the strength and challenges of the technological interventions used to support NICU staff-parents communication, we have categories them into three group. These are communication technologies used to provide 1. Neonatal status 2.parental education and 3.ad-hoc communication services. These classes of interventions are discussed in the next subsections.
\subsection{Communication Technologies used to Provide Neonatal Status}
Researchers have developed various systems  such as video-conferencing tools and Natural Language Generation (NLG) system which are deployed in the NICU to ensure that parents receive continuous support from staff  as well as updated  neonatal information even when they are not in the hospital. 

\textbf{Video-Conferencing Systems}: Virtual and online support system are currently being used in the NICU to provide specialized information regarding the care of the hospitalized infant, prognoses of these infants and the role and skills expected of parents at discharge \citep{Joshi2016, Luu2017, Yang2014}. These system not only  lessen parents' need to travel to the hospital, but they also provide a portal through which tailored information pertinent to the care of their newborn infants can be accessed at a time and location of choice. Gray et al. \citep{Gray2000a} deployed Baby CareLink tele-medicine application: a tool that provided information to parents using  both a website and video-conferencing system from the NICU. Their findings show that the application supported the educational and emotional needs of families. However, lack of access to the Internet by many parents poses logistic problem for the adoption of this kind of approach. These findings are similar to Linderberg and Ohrling \citep{Lindberg2012a} study who deployed a video-conferencing system to support parents as they take care of their infants back at home. Gund et al. \citep{Gund2013} identified that parent families preferred using skype  to acces infant health information rather than phone call. However the NICU nurses were reluctant and avoided using the tool because they found it hard to use. 

The government of Canada invested three million dollar to implement ChezNICU Home system to maintain virtual contact between parents  and their children when they are unable to be with them in the hospital \citep{McNutt2017}. The system strengthen staff-parent relationships and augment the care provided by enhancing communication and providing accessible, standardized, up-to-date education materials to parents. Joshi et al. \citep{Joshi2016} offered webcam camera to parents of preterm infants and used them for communication between staff and parents. However, their study reported that the web camera increased nurses' workload and stress, which they perceived as having an adverse effect on the ability to provide quality care. In addition, some parents preferred to visit their infants at the hospital instead of using the web camera to interact with the staff. 


\textbf{Natural language generation (NLG) systems}: These systems have been increasingly used for the creation of e-Health systems \citep{Hueske-kraus2003,Pauws2018, Lindahl2005}. Within healthcare, increasing amounts of patient data are being stored within computerised health databases. This information is being stored in patient records and is combined with drug databases and knowledge bases of medical terminology. Besides helping to provide information support to clinicians, NLG is playing a greater role in providing patients with access to information in a personal form. In NICU setting, BabyTalk family system \citep{Mahamood2011}, are being used to communicate medical information summaries for parents of preterm neonatal infants. This includes BT-45 system that automatically summarizes 45 minutes of continuous and discrete data in four stages, to assit in real-time decision making \citep{Gatt2009}. BT-Nurse \citep{Hunter2011} generates summaries for nurses, to assist in shift handover. This allowed the nurses to provide correct information to the parents which did not conflict with nurses in previous shifts.

The research and improvement of BT-family system is still in progress. Recently, the researchers are exploring the use of representing information in a emotion sensitive manner. This development has led to the rise of ‘Affective’ NLG (ANLG), which takes into account the emotional aspects of the parents and modifies its textual output  \citep{Mahamood2011}. The system has a document planner that generates a text structure in a more accessible narrative format for parents rather than producing technical diagnostic texts for nurses. After evaluating the system, they found out that the use of such affective strategies may be appropriate especially when communicating emotional sensitive information to parents with little knowledge about technologies.

However, these systems are still difficult to use when parents are not able to read the instruction in English. In addition, empirical testing of ANLG systems pose many challenges with very few past systems being tested. This makes it hard to determine the effectiveness of systems  because they can not be bench marked. 

\subsection{Ad-hoc day-to-day Communication Services}
 Ad hoc day-to-day communication when attending to infant care and treatment is essential to keep parents and NICU staff abreast with infant health status. Many of the ethical and medical issues that are encountered routinely in the neonatal unit are highly complex and have to be communicated to parents as soon as they occurs. 
 Globus et al. \citep{Globus2016} implemented the Short Messages Services (SMS) technique to provide daily update to parents regarding the health status of their preterm infants. They identified that use of SMS is an easy and user-friendly technology that delivered information to parents of hospitalized preterm infants. Parents were satisfied with the daily SMS updating, which resulted in better communication with the NICU staff. 
 
 This is similar to  Mburu et al. \citep{Mburu2018a} findings who identified that SMS were effective in sharing neonatal information with parents since they can be used in any geographical region and do not require parents to own expensive phone to access neonatal information. However, these communication services are not effective when parents do not respond immediately to the information shared by the NICU staff. For instance, Mburu et al. \citep{Mburu2018a} identified that parents often change their SIM card number thus they are not reachable when NICU staff require their consent in case of emergency in the unit. In addition, most family in low-income settings often share mobile phones in their households thus information sharing is ineffective when instant responds is expected from such parents.
 
 NICU staff also use phone calls for routine communication in the neonatal unit. For instance, Keeraan et al. \citep{Keraan2017} used phone calls to remind parents on the appointment scheduled for infants Retinopathy of prematurity (ROP) screening. Gosnell Family NICU \citep{UniversityofRochestermedicalSchool2019} allowed parents of hospitalized infants to call the unit whenever they needed information about their infants health status. In addition, the staff called the parents in case of any serious change in infant's condition. However, most parents did not understand the medical terms usually used by NICU staff and they had to refer to the medical term guide developed to assist them which eventually seemed to be a cumbersome process.
 

\subsection{Systems that Provide Parental Education}
The wide usage of smart phones in addition to affordable prices of wireless technologies has led to the explosion of mobile application developments. The increasing numbers of preterm infants being born globally, the high cost of caring for these infants and the need for frequent NICU communication push health care providers and system designers to create web and mobile application that can be used to provide parental education.These tools aim to complement staff-parents face-to-face communication. NICU-2-Home \citep{Garfield2014} and MyPreemie \citep{Doron2013} are mobile applications used to provide parental education information, monitoring of infants' progress and encouragement to parents of preterm infants. The applications proved that they can ease stress among parents and avoid the need of NICU staff from repeating information to them. They argued that smartphone application are practical for NICU parents shuttling for months between home, work, and the hospital with a desire for personalized and synchronized information.

Estrellita \citep{Hayes2014} and fitbaby \citep{Hayes2010} include two interfaces: a mobile application and a web-based clinical interface. The mobile applications, allows users to record observations of daily living (ODLs) for the infant and caregiver, share these data with clinical providers, and visualize past recorded ODL data. Through the website, healthcare providers can interact with the caregiver and keep abreast with the infant’s ODLs through a series of simple visualizations and data summaries. The system evaluation showed that  the systems were usable by parents with minimal training.

However, some of these aforementioned web and mobile application for NICU require users to own expensive devices and high broadband internet for the users to access the information. Most parents in these studies mentioned that lack of Internet access hindered them from accessing the parental information.  In addition, application such as MyPreemie, is for sale adding infant care cost to the already overstretched finances. According to Enweronu-Laryea et al. \citep{Enweronu-Laryea2018} economic cost of newborn hospitalized baby is high and available interventions required to support parents should be cost-effective.

In addition, similar to existing technologies used to support preterm infants' parents, parents were not involved in the design process. Health practitioners shared the systems' requirements with system developers who built the tools. Parents were only included during the evaluation process. This raises a question, how then can we fully involve parents of preterm infants in the design process of technological interventions to ensure it affordable and also able to meet their emotional needs? To answer this question, we acknowledge there are several factors that affect a productive design activity between researchers, health personnel and parents of hospitalized infants. These include: power imbalance which often emerges due to hierarchical relations in health environment, unavailability of parents and the sensitivity of the design topic. In the next section we discuss the aforementioned factors and provide recommendation of possible strategies of bypassing them.

\section {Hierarchical Relations in Health Environment}
Several researchers such as Guo and Hoe-Lian \citep{Guo2014}, Hampshire et al. \citep{Hampshire2015} and Farr \citep{Farr2017a} have used Foucauldiaon (discussed in section 2.2.1)lenses to understand power relations in Participatory design (PD) projects. In general, PD aims to place the control of knowledge in the hands of participants, empowering them to voice and negotiate their design ideas before arriving at a consensus. However, despite the ubiquitous commitment to  participation in PD research, there is a distinct lack of consensus about the meaning of "participation". To understand participation, Hampshire et al. \citep{Hampshire2015} categorized participation into four modes namely: contractual, consultative, collaborative and collegiate. In all these modes of participation, power exists only in its exercise, diffusing in all social relations. 

For instance Hampshire et al. \citep{Hampshire2015} explored power relations within a participatory health and social needs project and identified that shift in power balance  happened throughout the project. Although the shift was slow and limited, it was mainly influenced when participants have different knowledge capacity, agenda and position of power. Molina-Mula  et al. \citep{Molina2018} conducted a qualitative study to analyse the decision-making capabilities of patients from nurses’ perspectives of interprofessional relationships. The analysis of their results showed that patients are not effectively included in the communication channels that allow them to fully access information about their diseases. Consequently, this limit them from making autonomous decisions about their care. As a recommendation, they argue that health personnel should establish a relationship of equality with the patients to ensure patients and their families are not excluded from making decisions about their care. Wilson et al. \citep{Wilson2015} involved people living with Aphasia in the co-design process to explore appropriate co-design techniques required to fully engaged all participants in the design process.They identified the value of democracy in the design space and the need to empower underprivileged participants in the design process.They recommend the use of variety of techniques to encourage participation.

In NICU context, literature shows that the care of hospitalized infants is predominantly done by the staff, limiting the parental role to merely instruction receivers \citep{Jones2007a,Obeidat2009, Heidari2015a}. In most cases, parents; who are the primary caregiver of the admitted infants find it difficult to  express their views and make decision over infant's health in the midst of the health staff. A number of barriers to effective interaction between NICU staff and parents include inadequate or conflicting information, cultural influences and power inequality in infant's health decision making \citep{Jones2007a, VanMcCrary2014}. Obeidat et al. \citep{Obeidat2009}, Wigert et al \citep{Wigert2014b} and Ionio et al. \citep{Ionio2016} studies report that NICU parents feel powerless, hopeless and alienated within the NICU environment mainly because they lack the information required to involve them in the decision making of their infants' health.

In health environment participatory and co-design approach focus on empowering participants ensuring that they are equally engaged in the decision making of their healthcare \citep{Piper2018}. These approach are currently recommended to ensure that the patients / caregivers are involved in the decision making of their health care thus mitigating some of the sources of communication challenges in health environment \citep{Blandford2018,Dietrich2017, Birnbaum2016}. However, literature does not report the use of these approaches in NICU context. We therefore, ask, how can these approaches be explored to ensure we tap into parents potential to improve participation in NICU environment thus ensuring an effective design process that allows both NICU parent and staff to equally engage in decision making?

\section{Engaging Parents in the Participatory Design Process}
As discussed in section 2.5, the aforementioned NICU technological interventions have proven beneficial to parents of preterm infants. However, parents experience some challenges while using them which could have been avoided if they were included in the design process. Luck \citep{Luck2018} proposes that users need to be included in the design process to ensure that their needs and interests are based on real findings rather than assumptions. It is critical that users be actively involved in the design process since they will be ultimately affected by the final design. Based on these facts, it is therefore essential to engage both NICU staff and parents of hospitalized infants in the design  process of the tools meant to support NICU communication. 

According to Balaam et al. \citep{Balaam2013} designing with new parents especially the mothers, bring new challenges for participatory design methods. Researchers have to consider sensitive methods for engaging new mothers in participatory design that take account of the cognitive, emotional and physical limitations of mothers' new role. Moreover, mothers tend to focus more on their children rather than on the design activities.  For instance, D'Ignazio et al. \citep{Ignazio2016} involved a group of mothers and experts in the design process that focused on improving the design of breast pump. They identified that designing for the postpartum experience is complex and context-sensitive, as it sits at the intersection of numerous legal, political, social and cultural factors. Further,Balaam et al. \citep{Balaam2015} and Wardle et al. \citep{Wardle2018a} identified that mothers with young infants have limited time to engage in the design process and suggested that the design tasks need to be flexible, quick and undemanding to allow mother to fully participate.

 Motherhood is a complex life phase that brings with it physical changes, changing relationships, new responsibilities, and shifting notions of personal identity \citep{Gibson2013}. New mothers can experience social exclusion, elevated stress levels, and postnatal exhaustion in the early phases of parenting  \citep{Ignazio2016, Gibson2013}. In addition mothers face numerous social obstacle especially those in sensitive settings such as marginalized regions due to cultural and political factor as well as poor provision of health and social care \citep{Goodburn2015,Hardee2012}. In summary, D'Ignazio et al. \citep{Ignazio2016} reiterates that it is challenging to involve mothers in the design of technologies meant to improve the experience of motherhood. Therefore how can we effectively include mothers of young children in the design process on NICU communication intervention despite the numerous challenges that they face in their early stage of motherhood?

\section{Participatory Design in Sensitive Settings}
Literature has proved that conducting participatory design in sensitive settings poses more challenges for HCI researchers \citep{Hussain2012, Wadley2016, Garzotto2008}. This is mainly due to language barrier, low literacy levels, socio-economic and cultural barriers \citep{Hussain2012}. For this reason, Waycott et al. \citep{Waycott2015} coined the term "sensitive HCI" which refers to HCI research conducted in increasingly sensitive and difficult settings. Studies such as end of life care \citep{Borgstrom2017a}, bereavement  support \citep{Massimi2013}, design for mental health issues \citep{Wilson2015,Thieme2013} fit in the sensitive HCI research category. 

Research in these settings can produce complex ethical dilemmas that are often emergent, diverse, and highly contextualized. On the other hand, designing with emergent users help to uncover people's behaviours, motivations and goal in order to design for them. It also enable researchers to build relationships with participants in order to determine how they understand the technologies in their environment to improve resilience strategies \citep{Garzotto2008}. Such studies are therefore encouraged to promote a wider understanding of issues that arise in sensitive settings, to generate dialogue, foster shared learning, and promote reflexive practice.

However, few studies address the real-life challenges of doing sensitive HCI in developing world or how participatory design methods have to be adapted in such settings \citep{Winschiers-Theophilus2010, Oyugi2008b}. Hussain et al. \citep{Hussain2012} analysis shows that several studies that have used participatory design methods in developing countries do not report how participation was organized and whether the methods used to involve participants were successful or not. This is similar to \citep{Spinuzzi2005} work who observed that most HCI researchers label their work "participatory design" without being accountable to established participatory methods or techniques used. Winschiers, H. \citep{Winschiers2006} study proves that participatory design in a cross-cultural context goes beyond the involvement of users in the design of the product but should include an appropriation of the design process itself. 

Furthermore, participatory design  has its own methodological orientations, methods and techniques but these aspects can not be replicated from one project to another \citep{Danaher2013}. Participants have different needs and researchers need to identify the appropriate techniques for communicating abstract ideas to participants and those that enable participants to express their views and articulate their needs. According to Maunder et al. \citep{Maunder2007} and Hussain et al. \citep{Hussain2012}, most participants in marginalized settings have limited exposure to technology thus researchers need to alter or localise the design activities to improve participants knowledge and understanding of the technologies. Therefore, HCI researchers should set aside enough time to understand the participants and their environments before engaging in design activities. The best way to get insight is to spend time with different stakeholders, and not only the intended end users \citep{Hussain2012, Jones2008}.

However, most Western world researchers organize research in marginalized settings with underlying assumptions that communities are democratic with moderate literacy rates and that there is a reasonable technological infrastructure present \citep{Danaher2013,Jones2017}. Although these assumptions can also be questioned in Western projects, Puri et al. \citep{Puri2004} point out that it is unrealistic to make any of these assumptions in a developing country context. To avoid such assumptions Hussain et al. \citep{Hussain2012} and Winschiers, H. \citep{Winschiers2006} propose that researchers should familiarize with the local people and their culture and use this understanding when engaging with the participants. Using this approach encourages designer empathy, promotes user ownership and empowers marginalized individuals \citep{Mattson2014a}. Therefore, there is still a need for more in-depth analyses of case studies exploring both challenges and opportunities for conducting participatory design projects for marginalized people in developing countries.

In this research, we focus on engaging mothers of premature infants within the context of low-resource settings in South Africa, in the design process of a possible technological intervention that could augment the NICU communication. We opt to use mothers and not both parents to ensure that we do not eliminate single mothers in the study or even make them feel uncomfortable as they engage with other participants. As shown in literature, no previous research of the kind has been conducted in low-resource context. This provides a  great opportunity to investigate how NICU in low-resource context are currently using technologies to support communication, the challenges they encounter and the appropriate participatory design methods/techniques required to engage this vulnerable group of participants in a design process to ensure that they are able to voice their design ideas and co-design possible intervention that will meet their NICU communication needs. There is an opportunity  for generating grounded theory that described mothers experience as co-designers and how the design process empowered them using participatory techniques. These techniques, when paired with grounded theory methods, build testable theories from the ground up, based on the real experiences of all stakeholders involved in the design process. This approach offers practical solutions for other researchers conducting research in NICU context thus furthering a theoretical research agenda.

\section{Chapter Summary}
In this chapter we discussed the preterm birth statistics and its associated stress and financial burden. We address the communication challenges in the NICU,it attributing factors and some of the interventions used to alleviate NICU communication. We focus on technologies used to support NICU communication, analysing their strength and challenges. There is an emphasis in this chapter on the significance of engaging both NICU parents and staff in the design process of possible NICU communication technological intervention. We conclude the chapter by discussing factors that could affect participatory design in NICU context thus highlighting the design gap that we focus on filling in this study.The details regarding the methods used in this study are discussed in the next chapter, Chapter 3. 








 

% This just dumps some pseudolatin in so you can see some text in place.
%\blindtext
